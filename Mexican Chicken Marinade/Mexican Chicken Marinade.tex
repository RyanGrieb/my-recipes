\documentclass[14pt]{extarticle}
\usepackage[utf8]{inputenc}
% --- REMOVE PAGE NUMBERING ---
\pagestyle{empty}
% --- MARGIN SETUP ---
\usepackage[left=0.75in, right=0.75in, top=0.5in, bottom=0.5in]{geometry}
% --- LIST SETTINGS ---
\usepackage{enumitem}
\setlist[itemize]{leftmargin=1em}
\setlist[itemize]{leftmargin=1em, itemsep=0.2em, label={\raisebox{-0.1ex}{\large\textbullet}}}
\setlist[enumerate]{leftmargin=1.5em, label=\textbf{\arabic*.}, itemsep=0.25em}
\usepackage{parskip}
% --- CONTROL SECTION FONT SIZE ---
\usepackage{titlesec}
\titleformat{\section}{\large\bfseries}{}{0em}{}
\titleformat{\subsection}{\normalsize\bfseries}{}{0em}{}
\titlespacing*{\subsection}{0pt}{0.75em}{0.25em}
\titlespacing*{\section}{0pt}{1em}{0.5em}
% --- FONT SETUP ---
\usepackage{helvet}
\renewcommand{\familydefault}{\sfdefault}
% --- CUSTOM COMMAND FOR TITLE ---
\newcommand{\makeRecipeTitle}[1]{
    {\noindent \Large \textbf{#1}}
    \par \vspace{-0.75em}
}


\begin{document}

% ================= PAGE 1 =================
\makeRecipeTitle{Mexican Chicken Marinade}

\section*{Ingredients}

\begin{itemize}
      \item Chicken Thighs – 8
      \item Avocado Oil – 1/3 Cup
      \item Chili Powder/Flakes – 1 tsp
      \item Chipotle Paste/Sauce – 3 tsp (e.g., La Costeña)
      \item Cilantro – ½ Cup, chopped
      \item Cumin Powder – 1 tsp
      \item Garlic Powder – 2 tsp
      \item Limes – 2 (juiced)
      \item Onion Powder – 1.5 tbsp
      \item Oregano – 1.5 tbsp
      \item Paprika – 1 tsp
      \item Black Pepper – ½ tsp
      \item Salt/MSG – 1.5 tsp
\end{itemize}

\newpage
\makeRecipeTitle{Mexican Chicken Marinade}
\section*{Instructions}

\begin{enumerate}

      \item \textbf{Make Marinade} \\
            In a resealable bag or bowl, combine avocado oil, chili powder, chipotle paste, cilantro, cumin, garlic powder, lime juice, onion powder, oregano, paprika, pepper, and salt/MSG. Whisk or shake until dissolved and combined.

      \item \textbf{Marinate Chicken} \\
            Add the chicken thighs, ensuring they are well coated. Refrigerate for at least 30 minutes (or overnight for deeper flavor).

      \item \textbf{Sear} \\
            Preheat a cast iron pan to ripping hot heat. Remove chicken from marinade (discard excess). Sear on both sides for 30–60 seconds to get a char.

      \item \textbf{Cook} \\
            \textbf{Option A (Air Fryer):} Cook at 400$^\circ$F for 12 minutes, flip, then cook for 8 more minutes. \\
            \textbf{Option B (Oven):} Bake at 400$^\circ$F for 20–25 minutes.

      \item \textbf{Serve} \\
            Let the chicken rest for a few minutes before serving.

\end{enumerate}

\end{document}
