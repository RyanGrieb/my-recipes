% !TEX root = Charro Beans.tex
\documentclass[14pt]{extarticle}
\usepackage[utf8]{inputenc}
% --- REMOVE PAGE NUMBERING ---
\pagestyle{empty}
% --- MARGIN SETUP ---
\usepackage[left=0.75in, right=0.75in, top=0.5in, bottom=0.5in]{geometry}
% --- LIST SETTINGS ---
\usepackage{enumitem}
\setlist[itemize]{leftmargin=1em}
\setlist[itemize]{leftmargin=1em, itemsep=0.2em, label={\raisebox{-0.1ex}{\large\textbullet}}}
\setlist[enumerate]{leftmargin=1.5em, label=\textbf{\arabic*.}, itemsep=0.25em}
\usepackage{parskip}
% --- CONTROL SECTION FONT SIZE ---
\usepackage{titlesec}
\titleformat{\section}{\large\bfseries}{}{0em}{}
\titleformat{\subsection}{\normalsize\bfseries}{}{0em}{}
\titlespacing*{\subsection}{0pt}{0.75em}{0.25em}
\titlespacing*{\section}{0pt}{1em}{0.5em}
% --- FONT SETUP ---
\usepackage{helvet}
\renewcommand{\familydefault}{\sfdefault}
% --- CUSTOM COMMAND FOR TITLE ---
\newcommand{\makeRecipeTitle}[1]{
    {\noindent \Large \textbf{#1}}
    \par \vspace{-0.75em}
}


\begin{document}

% ================= PAGE 1 =================
\makeRecipeTitle{Charro Beans}

\section*{Ingredients}

\textbf{For the Beans}
\begin{itemize}[itemsep=0.1em, label={\raisebox{-0.1ex}{\large\textbullet}}]
    \item Dried Pinto Beans – 1 lb (454g), rinsed and sorted
    \item White Onion – ½ medium, peeled and halved
    \item Garlic Cloves – 6, peeled
    \item Water – 10 cups
    \item Kosher Salt – 2 tsp
\end{itemize}

\textbf{Charro Base \& Meats}
\begin{itemize}[itemsep=0.1em, label={\raisebox{-0.1ex}{\large\textbullet}}]
    \item Bacon – 8 oz, diced
    \item Cooked Ham – 1 cup, diced (optional)
    \item Mexican Chorizo – 4 oz, casings removed (optional)
    \item White Onion – 1 large, diced
    \item Garlic Cloves – 3, minced
    \item Jalapeño Peppers – 2 large, seeded and diced
    \item Roma Tomatoes – 3, diced
    \item Chipotle Pepper in Adobo – 1, minced (optional)
    \item Chicken Broth – ½ cup (for deglazing)
\end{itemize}

\textbf{Seasonings \& Finishing}
\begin{itemize}[itemsep=0.1em, label={\raisebox{-0.1ex}{\large\textbullet}}]
    \item Cumin – 2 tsp
    \item Mexican Oregano – ½ tsp
    \item Black Pepper – ½ tsp
    \item Fresh Cilantro – ½ cup, chopped
\end{itemize}

\newpage
\makeRecipeTitle{Charro Beans}
\section*{Instructions}

\begin{enumerate}

    \item \textbf{Cook the Beans (1.5--2 hours)} \\
          Place rinsed beans, onion halves, 6 peeled garlic cloves, water, and 2 tsp salt in a large Dutch oven or heavy pot. Bring to a boil over medium-high heat, then reduce to medium-low, cover, and simmer until beans are tender—about 1.5 to 2 hours. Check periodically and add water if needed to keep beans submerged. When beans are tender, remove and discard the onion halves and garlic cloves.

    \item \textbf{Create Bean Paste (5 minutes)} \\
          Remove pot from heat. Transfer 1 cup of cooked beans and 1 cup of the hot bean broth to a blender. \\
          \textit{\textbf{Safety Note:} When blending hot liquids, remove the center cap from the blender lid and cover the opening with a folded kitchen towel to allow steam to escape. Start on low speed and increase gradually.} \\
          Blend until completely smooth to create a creamy bean paste. Transfer the remaining beans and broth to a large heatproof bowl and set aside.

    \item \textbf{Render Bacon \& Cook Meats (12--15 minutes)} \\
          In the now-empty pot, cook diced bacon over medium heat, stirring occasionally, until fat renders and edges are crispy (8--10 mins). Remove bacon with a slotted spoon and set aside. \\
          \textbf{If using chorizo:} Add to bacon fat, breaking apart with a spoon. Cook until browned (4--5 mins). Remove and set aside. \\
          \textbf{If using ham:} Cook briefly until warmed (1--2 mins). Remove and set aside. \\
          Pour off all but 3 tablespoons of the rendered fat from the pot.

    \item \textbf{Sauté Aromatics (12--13 minutes)} \\
          Add diced onion to the bacon fat and sauté until softened (5 mins). Stir in minced garlic and jalapeño peppers; cook for 1 minute until fragrant. Add diced tomatoes and chipotle pepper (if using); cook 5--7 minutes, stirring occasionally, until tomatoes soften and release their juices.

    \item \textbf{Deglaze \& Combine (22--25 minutes)} \\
          Pour in ½ cup chicken broth and scrape the bottom of the pot with a wooden spoon to release flavorful browned bits. Let reduce slightly (1 min). \\
          Return all cooked meats to the pot. Add the reserved beans and their broth back to the pot. Stir in the bean paste and ½ cup water (swirl water in blender to rinse remaining paste). Add cumin, oregano, and black pepper. Bring to a gentle boil, then reduce heat to low and simmer, covered, for 15--20 minutes, stirring occasionally.

    \item \textbf{Finish \& Serve} \\
          Remove from heat and stir in chopped cilantro. Taste and add salt as needed, beginning with 1 teaspoon and adjusting from there (depending on saltiness of meats). Serve hot with your choice of garnishes.

\end{enumerate}

\end{document}